\documentclass[PMO,lsstdraft,authoryear,toc]{lsstdoc}
\input{meta}

% Package imports go here.
\usepackage{listings}
\usepackage{xcolor}
\usepackage{enumitem}
\usepackage{booktabs}
% Local commands go here.
% YAML language definition for listings
\lstdefinelanguage{yaml}{
  keywords={true,false,null,y,n},
  keywordstyle=\color{blue}\bfseries,
  basicstyle=\ttfamily\small,
  sensitive=false,
  comment=[l]{\#},
  commentstyle=\color{green!60!black}\itshape,
  stringstyle=\color{red},
  morestring=[b]',
  morestring=[b]",
  moredelim=[l][\color{orange}]:-,
  moredelim=[l][\color{magenta}]:-,
  morekeywords={apiVersion,kind,metadata,name,spec,type,securityContext,
    runAsNonRoot,runAsUser,fsGroup,seccompProfile,containers,allowPrivilegeEscalation,
    capabilities,readOnlyRootFilesystem,privileged,podSelector,policyTypes,ingress,egress}
}

% Configure code listings
\lstset{
    basicstyle=\ttfamily\small,
    breaklines=true,
    frame=single,
    numbers=left,
    numberstyle=\tiny,
    keywordstyle=\color{blue},
    commentstyle=\color{green!60!black},
    stringstyle=\color{red},
    showstringspaces=false,
    backgroundcolor=\color{gray!10},
    captionpos=b
}

%If you want glossaries
%\input{aglossary.tex}
%\makeglossaries

\title{Hardening Kubernetes Workload}

% This can write metadata into the PDF.
% Update keywords and author information as necessary.
\hypersetup{
    pdftitle={Hardening Kubernetes Workload},
    pdfauthor={Gonzalo Seriche},
    pdfkeywords={kubernetes, security, CIS benchmark, container security, workload hardening}
}

% Optional subtitle
% \setDocSubtitle{A subtitle}

\author{%
Gonzalo Seriche
}

\setDocRef{ITTN-079}
\setDocUpstreamLocation{\url{https://github.com/lsst-it/ittn-079}}

\date{\vcsDate}

% Optional: name of the document's curator
% \setDocCurator{The Curator of this Document}

\setDocAbstract{%
This technical note provides comprehensive guidance for implementing security controls and hardening measures for Kubernetes workloads based on CIS Benchmark recommendations. It covers pod security policies, network security, container image security, and compliance monitoring for on-premise Kubernetes deployments.
}

% Change history defined here.
% Order: oldest first.
% Fields: VERSION, DATE, DESCRIPTION, OWNER NAME.
% See LPM-51 for version number policy.
\setDocChangeRecord{%
  \addtohist{1}{2024-09-13}{Initial release.}{Gonzalo Seriche}
  \addtohist{1}{2024-12-19}{Add primary best practices and security guidelines}{Gonzalo Seriche}
}


\begin{document}

% Create the title page.
\maketitle
% Frequently for a technote we do not want a title page  uncomment this to remove the title page and changelog.
% use \mkshorttitle to remove the extra pages

% ADD CONTENT HERE
% You can also use the \input command to include several content files.

\section{Introduction}

This document outlines the security measures, implementation guidelines, and best practices for hardening Kubernetes workloads in accordance with the Center for Internet Security (CIS) Benchmark. The guidance provided here is specifically tailored for on-premise Kubernetes deployments and focuses on establishing a robust security posture while maintaining operational efficiency.

\section{Pod Security Standards}

\subsection{Pod Security Context Implementation}

Pod security context defines the privilege and access control settings that are essential for maintaining a secure Kubernetes environment. The following configurations establish the foundation for pod-level security:

\begin{lstlisting}[language=yaml]
apiVersion: v1
kind: Pod
metadata:
  name: secure-pod
spec:
  securityContext:
    runAsNonRoot: true
    runAsUser: 1000
    fsGroup: 3000
    seccompProfile:
      type: RuntimeDefault
\end{lstlisting}

The security context implementation enforces several critical security controls:

\begin{itemize}
    \item Non-root execution prevents privilege escalation scenarios
    \item Specific UID assignments ensure precise access control
    \item Filesystem group configurations maintain proper data access boundaries
    \item SecComp profiles limit available system calls to reduce the attack surface
\end{itemize}

\subsection{Container Level Security Controls}

Container security context configurations provide granular control over container execution parameters:

\begin{lstlisting}[language=yaml]
spec:
  containers:
  - name: secured-container
    securityContext:
      allowPrivilegeEscalation: false
      capabilities:
        drop: ["ALL"]
      readOnlyRootFilesystem: true
      privileged: false
\end{lstlisting}

These settings establish a defensive posture through:

\begin{itemize}
    \item Prevention of privilege escalation attempts
    \item Removal of unnecessary Linux capabilities
    \item Implementation of read-only root filesystems
    \item Explicit prohibition of privileged execution modes
\end{itemize}

\section{Network Security Framework}

\subsection{Default Network Policies}

Implementation of a default deny policy establishes a zero-trust networking baseline:

\begin{lstlisting}[language=yaml]
apiVersion: networking.k8s.io/v1
kind: NetworkPolicy
metadata:
  name: default-deny
spec:
  podSelector: {}
  policyTypes:
  - Ingress
  - Egress
\end{lstlisting}

\subsection{Granular Access Controls}

Specific allow rules should be implemented for required communications:

\begin{lstlisting}[language=yaml]
apiVersion: networking.k8s.io/v1
kind: NetworkPolicy
metadata:
  name: allow-specific
spec:
  podSelector:
    matchLabels:
      app: web
  ingress:
  - from:
    - podSelector:
        matchLabels:
          role: frontend
    ports:
    - protocol: TCP
      port: 80
\end{lstlisting}

\section{Resource Management and Quotas}

\subsection{Resource Quota Implementation}

Resource quotas prevent resource exhaustion and ensure fair resource allocation:

\begin{lstlisting}[language=yaml]
apiVersion: v1
kind: ResourceQuota
metadata:
  name: compute-resources
spec:
  hard:
    requests.cpu: "4"
    requests.memory: 8Gi
    limits.cpu: "8"
    limits.memory: 16Gi
\end{lstlisting}

\subsection{Default Resource Constraints}

Limit ranges establish default resource boundaries:

\begin{lstlisting}[language=yaml]
apiVersion: v1
kind: LimitRange
metadata:
  name: resource-constraints
spec:
  limits:
  - default:
      cpu: 500m
      memory: 512Mi
    defaultRequest:
      cpu: 200m
      memory: 256Mi
    type: Container
\end{lstlisting}

\section{Monitoring and Compliance Framework}

\subsection{Audit Logging Configuration}

Comprehensive audit logging enables security monitoring and compliance verification:

\begin{lstlisting}[language=yaml]
apiVersion: audit.k8s.io/v1
kind: Policy
rules:
- level: Metadata
  resources:
  - group: ""
    resources: ["pods", "services"]
- level: RequestResponse
  resources:
  - group: "apps"
    resources: ["deployments"]
\end{lstlisting}

\subsection{Compliance Monitoring Implementation}

The compliance monitoring framework should encompass:

\begin{itemize}
    \item Automated security scanning tools integration
    \item Continuous compliance status monitoring
    \item Policy enforcement automation
    \item Regular CIS Benchmark assessment procedures
\end{itemize}

\section{Security Best Practices}

\subsection{Image Security Controls}

Image security measures should include:

\begin{itemize}
    \item Mandatory image signing and verification
    \item Regular vulnerability scanning
    \item Base image update procedures
    \item Registry access control implementation
\end{itemize}

\subsection{Secret Management}

Implement robust secret management practices:

\begin{lstlisting}[language=yaml]
apiVersion: v1
kind: Secret
metadata:
  name: application-secrets
  annotations:
    vault.security.banzaicloud.io/vault-role: "app-role"
type: Opaque
data:
  APP_SECRET: <base64-encoded-secret>
\end{lstlisting}

\section{Implementation Guidelines}

\subsection{Deployment Process}

The security control implementation process should follow these phases:

\begin{enumerate}
    \item Security posture assessment
    \item Gap analysis against CIS Benchmark
    \item Control implementation prioritization
    \item Gradual security measure deployment
    \item Validation and testing
    \item Continuous monitoring establishment
\end{enumerate}

\subsection{Maintenance Procedures}

Establish regular maintenance procedures for:

\begin{itemize}
    \item Security patch management
    \item Configuration updates
    \item Policy refinement
    \item Compliance verification
\end{itemize}

\section{CIS Benchmark Compliance Testing}

\subsection{Kube-bench Implementation}

Kube-bench provides automated testing for Kubernetes CIS Benchmark compliance. The implementation varies based on Kubernetes versions and deployment methods.

\subsubsection{Version Compatibility}

Kube-bench supports multiple Kubernetes releases with specific test definitions:

\begin{lstlisting}[language=yaml]
# Job specification for Kubernetes 1.24+
apiVersion: batch/v1
kind: Job
metadata:
  name: kube-bench
spec:
  template:
    spec:
      hostPID: true
      containers:
        - name: kube-bench
          image: aquasec/kube-bench:v0.6.15
          command: ["kube-bench", "run", "--targets", "master,node"]
          volumeMounts:
            - name: var-lib-kubelet
              mountPath: /var/lib/kubelet
            - name: etc-systemd
              mountPath: /etc/systemd
            - name: etc-kubernetes
              mountPath: /etc/kubernetes
      restartPolicy: Never
      volumes:
        - name: var-lib-kubelet
          hostPath:
            path: "/var/lib/kubelet"
        - name: etc-systemd
          hostPath:
            path: "/etc/systemd"
        - name: etc-kubernetes
          hostPath:
            path: "/etc/kubernetes"
\end{lstlisting}

\subsubsection{Automated Scheduling}

Implement regular compliance testing through CronJobs:

\begin{lstlisting}[language=yaml]
apiVersion: batch/v1
kind: CronJob
metadata:
  name: kube-bench-schedule
spec:
  schedule: "0 1 * * *"  # Run daily at 1 AM
  jobTemplate:
    spec:
      template:
        spec:
          hostPID: true
          containers:
            - name: kube-bench
              image: aquasec/kube-bench:v0.6.15
              command: 
                - "kube-bench"
                - "run"
                - "--targets"
                - "master,node"
                - "--outputfile"
                - "/reports/compliance.json"
              volumeMounts:
                - name: var-lib-kubelet
                  mountPath: /var/lib/kubelet
                - name: reports
                  mountPath: /reports
          volumes:
            - name: var-lib-kubelet
              hostPath:
                path: "/var/lib/kubelet"
            - name: reports
              persistentVolumeClaim:
                claimName: compliance-reports-pvc
          restartPolicy: Never
\end{lstlisting}

\subsection{Version-Specific Configurations}

Different Kubernetes versions require specific considerations:

For Kubernetes 1.24 and newer:
\begin{itemize}
    \item Use kube-bench version 0.6.x
    \item Enable container runtime socket mounting
    \item Configure specific test suites for newer security features
\end{itemize}

For Kubernetes 1.23 and older:
\begin{itemize}
    \item Use compatible kube-bench versions (0.5.x)
    \item Adjust for legacy API versions
    \item Include deprecated security controls testing
\end{itemize}

\subsection{Results Processing}

Implement automated results processing:

\begin{lstlisting}[language=yaml]
apiVersion: v1
kind: ConfigMap
metadata:
  name: compliance-processor
data:
  process.sh: |
    #!/bin/bash
    REPORT_PATH="/reports/compliance.json"
    if [ -f "$REPORT_PATH" ]; then
      # Process results
      jq -r '.Controls[] | select(.status=="FAIL")' "$REPORT_PATH" > failures.json
      # Alert if necessary
      if [ -s failures.json ]; then
        # Send alerts
        curl -X POST ${ALERT_ENDPOINT} -d @failures.json
      fi
    fi
\end{lstlisting}

\section{Compliance Monitoring and Alerting Framework}

\subsection{Advanced Results Processing}

The compliance results processing system implements a comprehensive pipeline for analyzing kube-bench outputs and generating actionable insights. The system processes both JSON and YAML outputs, correlates findings across multiple clusters, and maintains a historical compliance database.

\begin{lstlisting}[language=yaml]
apiVersion: v1
kind: ConfigMap
metadata:
  name: compliance-processor-config
data:
  config.yaml: |
    processing:
      output_format: json
      retention_days: 90
      minimum_severity: MEDIUM
      excluded_checks:
        - 1.2.1  # Document exclusion rationale
        - 2.1.3  # Alternative control implemented
    alerting:
      threshold_critical: 85
      threshold_warning: 95
      notification_channels:
        - slack
        - email
        - pagerduty
    reporting:
      format: html
      schedule: "0 8 * * 1"  # Weekly reports
      recipients: ["security-team@org.com"]
\end{lstlisting}

\subsection{Monitoring Integration}

The monitoring framework integrates with existing observability platforms through a comprehensive metrics pipeline:

\begin{lstlisting}[language=yaml]
apiVersion: monitoring.coreos.com/v1
kind: ServiceMonitor
metadata:
  name: compliance-metrics
spec:
  selector:
    matchLabels:
      app: kube-bench
  endpoints:
  - port: metrics
    interval: 30s
    path: /metrics
  - port: compliance
    interval: 5m
    path: /compliance
---
apiVersion: monitoring.coreos.com/v1
kind: PrometheusRule
metadata:
  name: compliance-alerts
spec:
  groups:
  - name: compliance.rules
    rules:
    - alert: ComplianceScoreDegraded
      expr: compliance_score < 85
      for: 1h
      labels:
        severity: critical
      annotations:
        summary: Cluster compliance score below threshold
        description: Cluster {{ $labels.cluster }} compliance score is {{ $value }}
    - alert: HighSeverityFinding
      expr: compliance_findings{severity="HIGH"} > 0
      for: 5m
      labels:
        severity: critical
\end{lstlisting}

\subsection{Advanced Alerting Configuration}

The alerting system implements a sophisticated notification framework with different severity levels and escalation paths:

\begin{lstlisting}[language=yaml]
apiVersion: notification.toolkit.fluxcd.io/v1beta1
kind: Alert
metadata:
  name: compliance-alerts
spec:
  eventSeverity: info
  eventSources:
    - kind: Kustomization
      name: '*'
  providerRef:
    name: slack
---
apiVersion: notification.toolkit.fluxcd.io/v1beta1
kind: Provider
metadata:
  name: slack
spec:
  type: slack
  channel: security-alerts
  address: https://hooks.slack.com/services/YOUR-WEBHOOK-URL
  secretRef:
    name: slack-url
---
apiVersion: v1
kind: ConfigMap
metadata:
  name: alert-templates
data:
  critical.tmpl: |
    *Critical Compliance Alert*
    Cluster: {{ .Cluster }}
    Check: {{ .CheckID }}
    Description: {{ .Description }}
    Remediation: {{ .Remediation }}
  warning.tmpl: |
    Warning: Compliance Issue Detected
    Cluster: {{ .Cluster }}
    Details: {{ .Details }}
\end{lstlisting}

\subsection{Compliance Reporting and Analytics}

The reporting system generates comprehensive compliance analytics and trend analysis:

\begin{lstlisting}[language=yaml]
apiVersion: batch/v1
kind: CronJob
metadata:
  name: compliance-report-generator
spec:
  schedule: "0 1 * * 1"  # Weekly on Monday at 1 AM
  jobTemplate:
    spec:
      template:
        spec:
          containers:
          - name: report-generator
            image: compliance-reporter:v1
            env:
            - name: REPORT_PERIOD
              value: "7d"
            - name: INCLUDE_TRENDS
              value: "true"
            - name: REPORT_FORMAT
              value: "html,pdf"
            volumeMounts:
            - name: report-output
              mountPath: /reports
          volumes:
          - name: report-output
            persistentVolumeClaim:
              claimName: report-storage
\end{lstlisting}

\appendix
% Include all the relevant bib files.
% https://lsst-texmf.lsst.io/lsstdoc.html#bibliographies
\section{References} \label{sec:bib}
\renewcommand{\refname}{} % Suppress default Bibliography section
\bibliography{local,lsst,lsst-dm,refs_ads,refs,books}

% Make sure lsst-texmf/bin/generateAcronyms.py is in your path
\section{Acronyms} \label{sec:acronyms}
\addtocounter{table}{-1}
\begin{longtable}{p{0.145\textwidth}p{0.8\textwidth}}\hline
\textbf{Acronym} & \textbf{Description}  \\\hline

DM & Data Management \\\hline
\end{longtable}

% If you want glossary uncomment below -- comment out the two lines above
%\printglossaries





\end{document}